\documentclass{APA}

\begin{document}
\TITLE{Project Proposal - Survival in the Titanic}{Project Proposal - Group 2}{Maeki Kashana, Miguel Melo Ochoa, Alex Hayet, Francisco Gomez}{\today}{San Diego State University}{CS549 - Machine Learning}{Professor Xin Zhang}
\tableofcontents

\section{Problem Statement}
\cite{titanic}
The problem for this projnjnjnject is the titanic
\newpage



\section{Dataset Description}
	For this project, the dataset that we will be using is the Titanic 
\newpage



\section{Planned Methodology}
a
\newpage



\section{Evaluation Metrics \& Expected Outcomes}
	When looking at the evaluation metrics of the Titanic problem, the problem uses binary classification in order to show if a passenger survives the tragedy. In this instance, if a passenger is marked with a 1 this means they survived and a 0 would mean that they did not survive. When wanting to make a prediction on whether or not a passenger will survive there are a few metrics that we could use in order to find out the most likely outcome for the passenger.
	The first metric to look at would be the gender of the passengers. When looking at the data it seems that female passengers had a much larger survival rate than men passengers. Only about 20\% of men survived aboard the Titanic while around 75\% of women survived. This is largely due to the fact that women and children were the number one priority to get on the lifeboats first. Passenger class is another metric that affects the survival rate of passengers. First class passengers were more likely to survive in the event of sinking due to being higher up in the ship than second or third class passengers. Age is another factor that helped in the survival rate for passengers on the Titanic. As stated earlier, women and children were the highest priority passengers to get onto the lifeboats. This had a large impact on the survival rate for children aboard the Titanic and gave them a better chance at surviving. The ports in which passengers had boarded the Titanic from do show a correlation to survival. This is a socio-economic factor that does link back to the class of passengers as we can see a connection between the ports passengers boarded from to a class level on the boat. For example, Cherbourg has a lot of passengers that boarded into the first class.which means that they had a higher chance of surviving than passengers that boarded from Queenstown which had more third class passengers than Cherbourg. These are just some of the metrics that can be used to find expected outcomes in the event of the Titanic.

\newpage



\section{Work Distribution}

\subsection{Maeki Kashana}

\subsection{Miguel Melo Ochoa}

\subsection{Alex Hayet}

\subsection{Francisco Gomez}

\printbibliography
\end{document}
